% Copyright (c) 2011 University of Pennsylvania. All rights reserved.
% See http://www.rad.upenn.edu/sbia/software/license.html or COPYING file.
%
% Contact: SBIA Group <sbia-software at uphs.upenn.edu>

\documentclass[a4paper,12pt]{article}

\usepackage{cite}
\usepackage{geometry}
\usepackage{graphicx}
\usepackage{subfigure}

% Title Page
\title{Multiplicative Intrinsic Component Optimization \\ Software Manual}
\author{Andreas Schuh, Chunming Li}

\begin{document}
\maketitle
\tableofcontents
\setcounter{tocdepth}{1}

\pagebreak

% ============================================================================
% Introduction
% ============================================================================
\section{Introduction}
\label{intro}

This manual describes the usage of the Multiplicative Intrinsic Component\linebreak
Optimization (MICO) software. The MICO software is an implementation
of an algorithm for simultaneous bias field estimation and segmentation of
three-dimensional (3D) MR brain images~\cite{li,cvpr}. It can be used for:
\begin{itemize}
\item Bias correction.
\item Segmentation of independent patient scans.
\item Segmentation of longitudinal series of brain images.
\end{itemize}

The segmentations contain different labels for cerebrospinal fluid (CSF), \linebreak
gray matter (GM), white matter (WM), and the value zero for background (BG).
The bias field is estimated by a linear combination of a number of smooth basis
functions, which are constructed as 20 polynomials up to the third degree.

MICO can be used on the command-line using the \texttt{mico} command
or in C++ code by linking to the \texttt{mico} library. Note that only the most
important options of the \texttt{mico} command are documented in the following.
For more details on the available options, run the command with the option
\texttt{-{-}help}. This will output a brief description together with a list
of required and optional arguments. For a list of available options only,
use \texttt{-{-}helpshort} instead. If you want to use MICO directly in your
C++ code, please refer to the API documentation.

% ============================================================================
% Preparation
% ============================================================================
\section{Preparation}
\label{data}

Any input patient data has to be skull stripped first. This can be done using
\texttt{bet}~\cite{bet}, for example. In case of longitudinal studies,
all time points must further be co-registered across time using a rigid
registration with six degrees of freedom. Therefore, FSL's
\texttt{flirt}~\cite{flirt} command is commonly applied.
Please refer to the documentation of the named third-party tools for further
information on how they are applied.

Note that the input images must be three-dimensional images with scalar voxel
type of either DT{\_}UNSIGNED{\_}CHAR, DT{\_}SIGNED{\_}SHORT, DT{\_}SIGNED{\_}INT,
or DT{\_}FLOAT, and stored in the ANALYZE 7.5 or the NIfTI-1 file format.

\pagebreak

For your convenience, we included 11 scans of a single subject referred to
as \texttt{BC} which were acquired at different time points.
The directory where these files are installed is referred to in the following
as \texttt{example} shell variable. Further, we assume that the path to the
\texttt{mico} command is in your \texttt{PATH} environment variable.
\footnotesize
\begin{verbatim}
$ prefix=/top/directory/of/MICO/installation
$ setenv PATH "${prefix}/bin:${PATH}"
$ example=${prefix}/share/sbia/mico/example
\end{verbatim}
\normalsize
Or in case of BASH:
\footnotesize
\begin{verbatim}
$ prefix=/top/directory/of/MICO/installation
$ export PATH="${prefix}/bin:${PATH}"
$ example=${prefix}/share/sbia/mico/example
\end{verbatim}
\normalsize

% ============================================================================
% 3D Segmentation
% ============================================================================
\section{Segmentation of MR Brain Images}
\label{3d}

In order to segment a given MR brain image into CSF, GM, WM, and BG, simply
run the MICO command with the file path of the input image as argument, i.e.,
\footnotesize
\begin{verbatim}
$ mico ${example}/BC01-T1-byte_cbq
\end{verbatim}
\normalsize
The resulting segmentation (label image) will in this case be written to the
file \linebreak \texttt{BC01-T1-byte{\_}cbq{\_}segments.nii.gz} located in the
current working directory.
Use the \texttt{-{-}outputdir} option to specify a different output directory
and \texttt{-{-}suffix} to change the file name suffix of the
segmentation image, optionally including a different NIfTI-1 file format
extension. Moreover, you can specify more than one input image to segment.
Note, however, that each input image is segmented independently.
\footnotesize
\begin{verbatim}
$ mico --outputdir segmentations --suffix .hdr
        ${example}/BC01-T1-byte_cbq
        ${example}/BC02-T1-byte_cbq_flirtedToY16dof
        ${example}/BC03-T1-byte_cbq_flirtedToY16dof
        ${example}/BC04-T1-byte_cbq_flirtedToY16dof
        ${example}/BC05-T1-byte_cbq_flirtedToY16dof
        ${example}/BC06-T1-byte_cbq_flirtedToY16dof
        ${example}/BC07-T1-byte_cbq_flirtedToY16dof
        ${example}/BC08-T1-byte_cbq_flirtedToY16dof
        ${example}/BC09-T1-byte_cbq_flirtedToY16dof
        ${example}/BC10-T1-byte_cbq_flirtedToY16dof
        ${example}/BC11-T1-byte_cbq_flirtedToY16dof
\end{verbatim}
\normalsize
Alternatively, you can list the image file paths in a text file, one file path
per line, and use the \texttt{--inputlist} option instead, i.e.,
\footnotesize
\begin{verbatim}
$ mico --inputlist ${example}/BC.lst --outputdir segmentations --suffix .hdr
\end{verbatim}
\normalsize
Either command stores the segmentations in the two file NIfTI-1 image file
format under the same file name as the corresponding input image, but in the
subdirectory \texttt{segmentations} of the current working directory.

\pagebreak

% ============================================================================
% 4D Segmentation
% ============================================================================
\section{Segmentation of Longitudinal Study}
\label{4d}

In Section~\ref{3d} we have demonstrated how the \texttt{mico} command can
be used to segment 3D brain images into the three major tissue classes. This,
however, is done independently, i.e., no longitudinal information given
across time is used to maintain consistency among the segmentations of the
different time points. In order to take the longitudinal information into
consideration, use the option \texttt{--4d}. To segment the longitudinal
example study, run the command:
\footnotesize
\begin{verbatim}
$ mico --4d --outputdir segmentations --suffix _labels
        ${example}/BC01-T1-byte_cbq
        ${example}/BC02-T1-byte_cbq_flirtedToY16dof
        ${example}/BC03-T1-byte_cbq_flirtedToY16dof
        ${example}/BC04-T1-byte_cbq_flirtedToY16dof
        ${example}/BC05-T1-byte_cbq_flirtedToY16dof
        ${example}/BC06-T1-byte_cbq_flirtedToY16dof
        ${example}/BC07-T1-byte_cbq_flirtedToY16dof
        ${example}/BC08-T1-byte_cbq_flirtedToY16dof
        ${example}/BC09-T1-byte_cbq_flirtedToY16dof
        ${example}/BC10-T1-byte_cbq_flirtedToY16dof
        ${example}/BC11-T1-byte_cbq_flirtedToY16dof
\end{verbatim}
\normalsize
Alternatively, you can list the image file paths in a text file, one file path
per line, and use the \texttt{--inputlist} option instead, i.e.,
\footnotesize
\begin{verbatim}
$ mico --4d --inputlist ${example}/BC.lst --outputdir segmentations
        --suffix _labels
\end{verbatim}
\normalsize

% ============================================================================
% Bias Correction of MRI Brain Images
% ============================================================================
\section{Bias Correction of MRI Brain Images}
\label{bc}

The MICO algorithm performs a bias field estimation simultaneously with the
segmentation. Hence, the bias correction is always applied even if only the
final segmentations are saved to disk. In order to save also the bias corrected
image(s), use the \texttt{-{-}bias-correct} option. If only the bias correction
should be performed, use the option \texttt{-{-}bias-correct-only} instead.

For example, to perform a bias correction of the baseline image of the given
longitudinal example study, run the command:
\footnotesize
\begin{verbatim}
$ mico --bias-correct-only ${example}/BC01-T1-byte_cbq
\end{verbatim}
\normalsize
This will output the bias corrected image and save it in the current working \linebreak
directory as the file \texttt{BC01-T1-byte{\_}cbq{\_}biascorrected.nii.gz}.
If you want to use a different file name suffix or file format, use the
\texttt{--bias-correct-suffix} option to specify a different suffix for the
file names of the bias corrected images.

% ============================================================================
% Notes
% ============================================================================
\section{Notes}
\label{notes}

\begin{itemize}

\item Recognized file format extensions are
\texttt{.hdr} or  \texttt{.img} for header and data NIfTI-1 or ANALYZE 7.5 image
file pairs, \texttt{.hdr.gz} and \texttt{.img.gz} for compressed header and
data NIfTI-1 image file pairs, \texttt{.nii} for uncompressed NIfTI-1 images,
and \texttt{.nii.gz} for compressed NIfTI-1 images. Only if an input image was
stored in the ANALYZE 7.5 format and no particular format has been specified for
the output images, these output images are stored in the ANALYZE 7.5 format as
well.

\item Increase the weight for a certain tissue class if it is over segmented,
and decrease the weight if it is under segmented. The default weights used by
\texttt{mico} are reported in the help output of this command, i.e.,
\texttt{mico --help}.

\item If there is almost no intensity inhomogeneity in the input image,
a large weight (e.g. 10) can be used for the fuzzy C-means (FCM) term, the
parameter \texttt{-{-}lambda} of \texttt{mico}. In this case, the performance
of the MICO algorithm is close to the performance of the FCM algorithm.
If the intensity inhomogeneity is significant, however, set the weight of the
FCM term to zero or a very small positive number (e.g. 0.00001).
Typically, for 3T MR images, this parameter can be set to zero.
For 1.5T MR images, this parameter can be set to 1 if the intensity
inhomogeneity is not strong, and to 10 if there is almost no inhomogeneity.

\item By default, only the label map with the labels of the segmented structures
is output. The output of the membership images, one for each segmented structure,
can be requested using the \texttt{-{-}fuzzy} option. These images are commonly
also referred to as fuzzy segmentation(s).
\end{itemize}

% ============================================================================
% Bibliography
% ============================================================================
\bibliographystyle{splncs03}
\bibliography{references}

\end{document}          
